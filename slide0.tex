\documentclass[11pt,dvipdfmx,cjk]{beamer}
% パッケージの追加
\usepackage{graphicx} % 図を表示するためのパッケージの追加
\usepackage{bm} % 太字イタリックを使えるようにする 

% beamer に関する設定
% Adobe Reader の文字化けを防ぐおまじない
\AtBeginDvi{\special{pdf:tounicode EUC-UCS2}}
%テーマの指定
\usetheme{Singapore}
\renewcommand{\kanjifamilydefault}{\gtdefault} % 日本語をゴシック体にする
%\mathversion{bold} % これを指定するとデフォルトの数式が太字になる
%\usefonttheme{structurebold}

%--- colorblock 環境の定義
\setbeamercolor{upcolor}{fg=black,bg=blue!40} % ヘッダー部の色指定
\setbeamercolor{lowcolor}{fg=black,bg=blue!20} % body部の色指定
\newenvironment{colorblock}%
{\begin{beamerboxesrounded}[upper=upcolor,lower=lowcolor,shadow=true]}%
{\end{beamerboxesrounded}}%

% 本文のタイトルなど
\section{タイトル}
\

\begin{document}

\frame{\titlepage}
\frame{
  \frametitle{サポートベクトルマシン(SVM)}
  
}
\begin{frame}
  \frametitle{ハイパーパラメータ最適化(HPO)}
  \begin{itemize}
    \item 機械学習の性能を最大限に発揮するには適切な\\ハイパーパラメータの設定が必要不可欠
    \item 手動では経験的に決めることが多く時間のかかる作業
     \begin{itemize}
       \item 自動で調節する研究が行われている
     \end{itemize}
    \item 一般的に計算量が大きいため効率的な探索が必要
    \item 離散値、連続値、カテゴリ変数など様々な値を扱う必要がある
  \end{itemize}
\end{frame}
\begin{frame}
  \frametitle{Artificial Bee Colony(ABC)アルゴリズム}
  \begin{itemize}
    \item 蜂の採餌行動に着目した最適化アルゴリズム\footnote{Karaboga, Dervis. An idea based on honey bee swarm for numerical optimization. Vol. 200. Technical report-tr06, Erciyes university, engineering faculty, computer engineering department, 2005.}
    \item 働き蜂、追従蜂、偵察蜂の三種類の蜂によって各個体(食物源)の探索を行い、最適解を求める
    %\item 連続値の最適化を前提としている
    \item ABC自体の設定パラメータは少ない
  \end{itemize}
\end{frame}
\frame{
  \frametitle{先行研究におけるSVMのハイパーパラメータ最適化}
  \begin{itemize}
    \item ABCアルゴリズムを使用
    \item カーネル関数をガウスカーネルに固定
    \item 最適化したハイパーパラメータ
    \begin{itemize}
      \item SVMの$C$
      \item ガウスカーネルの$\gamma$
    \end{itemize}
    %式書いたほうがいいかも
  \end{itemize}
}
\section{問題点}
\frame{
  \frametitle{カーネル関数を固定することの問題点}
  
  \begin{itemize}
    \item カーネル関数はSVMの性能に大きな影響を与える
   \item ハイパーパラメータ空間の探索範囲が限定的
  \begin{itemize}
    \item カーネル関数が変わるとハイパーパラメータの数も変わる
  \end{itemize}
\end{itemize}
}

\section{提案手法}
\frame{
  \frametitle{提案手法}
  \begin{itemize}
    \item 4つのカーネル関数とそのハイパーパラメータも最適化対象とすることでより広いハイパーパラメータ空間を探索する 
    \item ABCアルゴリズムにおける解表現は以下のようにする
    \begin{align*}
      \text{解表現:(カーネル関数, C, gamma, coef0, degree)}
    \end{align*}
    \item gamma, coef0, degreeの3つはカーネル関数がもつハイパーパラメータでありカーネル関数によって取捨選択する 
    \item カーネル関数の更新はランダムに選ばれた個体とのルーレット選択  
  \end{itemize}
  \begin{align*}
    P = \frac{f(x_j)}{f(x_i)+f(x_j)}  
    \end{align*}
    \begin{align*}
    \text{$i$:更新個体~~$j$:ランダムに選ばれた値}
  \end{align*}
}


\end{document}